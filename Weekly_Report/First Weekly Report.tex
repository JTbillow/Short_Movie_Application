\documentclass[12pt, letterpaper]{article}
\usepackage[utf8]{inputenc}
\usepackage{enumerate}
\usepackage{comment}
\usepackage{cite}

%Title
\title{First Weekly Report}
\author{Team05}
\date{11.4.2018}

\begin{document}

\begin{titlepage}
\maketitle
\end{titlepage}

This week is week 7 for this semester (11.4.2018). For this week we focus on the interview with main stakeholder and do some research for the development of our short movie application.

\begin{enumerate}[1]
	\item What we have done in this week
		\begin{enumerate} [(a)]
			\item Research
				\begin{enumerate}[i]
					\item Market Research\\
					In this week, we do a questionnaire for the application’s users to find their attitude for fee, movies etc. It will affect our UE design in the future
					\item Technical Research\\
						\begin{enumerate}[1]	
							\item There are 3 kinds of application for IOS and Android software development: Web App, Hybrid App, and Native App.	\\

Native App: The native app for Android is programmed using the Java programming language. Application developed in this way runs directly on the system.\cite{viswanathan2014pros}\\

Web App: A web-app is a program which runs in a web browser. It is programmed using HTML, CSS, and JavaScript or other web-application frameworks.\cite{serrano2013mobile}\\

Hybrid App: A hybrid app combines the elements of both, native app and web app. It is coded using web languages (HTML, CSS, JavaScript), that can be ported to any platform inside a native container.\cite{serrano2013mobile}\\
						\begin{tabular}{|c|c|c|c|}
						\hline
						 &Web App&Hybrid App&Native App\\
						\hline
						Cost&Low&Medium&High\\
						\hline
						Maintenance&Simple&Simple&Hard\\
						\hline
						UE&Bad&Good&Good\\
						\hline
						Recognition for store&No&Yes&Yes\\
						\hline
						Install&No&Yes&Yes\\
						\hline
						Cross-platform&Good&Good&Bed\\
						\hline
						Cost&Low&Medium&High\\
						\hline
						\end{tabular}

						\item Tools and Language\\
Android: Android Studio/ Eclipse + ADT + Android SDK\\
         Java, HTML, CSS, JavaScript\\

IOS: Xcode\\
		Objective-C, C language, C++\\

Management: Git and GitHub\\
							\item Possible algorithm\\
Deep Neural Network Algorithm\\
In this application, we want to do “Recommended”, “Suggested”, “Related”, “Search”, “MetaScore” options like YouTube. Therefore, there should be an AI algorithm to judge that which kind of movies should be recommended to audience. Deep Neural Network Algorithm is a good way to finish “YouTube” algorithm.\cite{covington2016deep}


						\end{enumerate}
				\end{enumerate}
			\item Interview\\
For this interview, we have 4 parts for the main stakeholder interview: Introduction, Understanding, Questions and Equipment. The aim of interview is getting more details about requirement. But the interview data is 5th Nov. So, the result of this interview will be show on next weekly report.
			\item Outline for interim report
\begin{itemize}
\item [*]Introduction
\item [*]Background (Research)
\item [*]Requirements Specification
\item [*]Design and User Interface
\item [*]Project Management
\item [*]Future work
\end{itemize}
		\end{enumerate}		
	\item What we will do for next week.
		\begin{enumerate}[(a)]
		\item Minutes for formal meeting and stakeholder interview.
		\item Further improve the SRS document
		\item Do some UI design draft
		\item Using MindLine/Xmind draw Mind map

		\end{enumerate}
\end{enumerate}

\bibliographystyle{abbrv}
\bibliography{weekly_report}


\end{document}